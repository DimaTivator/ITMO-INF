% \documentclass[a4paper,14pt]{article}

% \usepackage[utf8]{inputenc}
% \usepackage[russian]{babel}
% \usepackage{caption}
% \usepackage{float}
% \usepackage[left=2cm,right=2cm,top=2cm,bottom=3cm]{geometry}
% \usepackage{multicol}
% \usepackage{multirow}
% \usepackage{tikz}
% \usepackage{array}

% \usepackage{amsmath}

% \begin{document}
    
\begin{flushleft}
    \textit{Таблица 1}
\end{flushleft}

\noindent\begin{tabular}{|c|c|c|c|} 
    
    \hline 
    Название интервала & Чистый интервал & 
    \multicolumn{2}{c|}{Интервал на рояле} \\
    \hline
    \begin{flushleft}
        малая терция
    \end{flushleft}
         & 6 : 5 = 1,2 & 1,1892... & $q^3$ \\
    \hline
    \begin{flushleft}
        большая терция 
    \end{flushleft}
        & 5 : 4 = 1,25 & 1,2599... & $q^4$ \\
    \hline 
    \begin{flushleft}
        кварта     
    \end{flushleft}
    & 4 : 3 = 1,333... & 1,3348... & $q^5$ \\
    \hline
    \begin{flushleft}
        квинта
    \end{flushleft}
    & 3 : 2 = 1,5 & 1,4983... & $q^7$ \\
    \hline 
    \begin{flushleft}
        малая секста     
    \end{flushleft}
    & 8 : 5 = 1,6 & 1,5874... & $q ^ 8$ \\
    \hline 
    \begin{flushleft}
        большая секста     
    \end{flushleft}
    & 5 : 3 = 1,666... & 1,6817... & $q^9$ \\
    \hline 
    \begin{flushleft}
        октава     
    \end{flushleft}
    & 2 : 1 = 2 & 2 & $q^{12}$ \\
    \hline
     
\end{tabular}
    
% \end{document}
